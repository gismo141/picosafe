\documentclass[
%	pdftex,%              PDFTex verwenden
	a4paper,%             A4 Papier
	twoside,%             Zweiseitig/Einseitigtotocnumbered,%    Literaturverzeichnis nummeriert einf�gen
%	idxtotoc,%            Index ins Verzeichnis einf�gen
	index=totoc,
%	halfparskip,%         Europ�ischer Satz mit abstand zwischen Abs�tzen
	parskip=half,
 	chapterprefix,%       Kapitel anschreiben als Kapitel
	headsepline,%         Linie nach Kopfzeile
%	footsepline,%         Linie vor Fusszeile
	11pt,%                Gr��ere Schrift, besser lesbar am bildschrim
	BCOR5mm,%							5mm Abstand Rand
	fleqn,%								linksb�ndig abgesetzte Formeln
  openany%              keine leeren seiten zwischen kapiteln
]{scrbook}

\usepackage{fancyhdr}
\pagestyle{fancy}
\rfoot{embedded projects}
\cfoot{\thepage}
\renewcommand{\headrulewidth}{0.4pt}
\renewcommand{\footrulewidth}{0.4pt}


\usepackage[english]{babel}
\usepackage{setspace}
\usepackage{a4wide}					 				%Bessere Ausnutzung der DinA4 gr��e ohne Randbemerkungen, auch a4 m�glich
\usepackage[T1]{fontenc}						%Europ�ischer Schriftsatz
\usepackage[utf8]{inputenc}
\usepackage{graphicx}								%Einbindung von Graphiken
\usepackage{amssymb,amsmath}				%Mathe und Schriftsatz
%\usepackage{amsfonts}
\usepackage{prettyref}
\usepackage{xspace}
\usepackage{colortbl}
\usepackage[format=plain,margin=1.5cm,small]{caption} %Legenden: mergin versetzt beitseitig nach innen, format=plain verhindert einr�cken (einfacher blocksatz)

\usepackage{listings}
\usepackage{color}
 
\definecolor{dkgreen}{rgb}{0,0.6,0}
\definecolor{gray}{rgb}{0.5,0.5,0.5}
\definecolor{mauve}{rgb}{0.58,0,0.82}
 
\lstset{ %
  language=Octave,                % the language of the code
  basicstyle=\footnotesize,           % the size of the fonts that are used for the code
  numbers=left,                   % where to put the line-numbers
  numberstyle=\tiny\color{gray},  % the style that is used for the line-numbers
  stepnumber=2,                   % the step between two line-numbers. If it's 1, each line 
                                  % will be numbered
  numbersep=5pt,                  % how far the line-numbers are from the code
  backgroundcolor=\color{white},      % choose the background color. You must add \usepackage{color}
  showspaces=false,               % show spaces adding particular underscores
  showstringspaces=false,         % underline spaces within strings
  showtabs=false,                 % show tabs within strings adding particular underscores
  frame=single,                   % adds a frame around the code
  rulecolor=\color{black},        % if not set, the frame-color may be changed on line-breaks within not-black text (e.g. comments (green here))
  tabsize=2,                      % sets default tabsize to 2 spaces
  captionpos=b,                   % sets the caption-position to bottom
  breaklines=true,                % sets automatic line breaking
  breakatwhitespace=false,        % sets if automatic breaks should only happen at whitespace
  title=\lstname,                   % show the filename of files included with \lstinputlisting;
                                  % also try caption instead of title
  keywordstyle=\color{blue},          % keyword style
  commentstyle=\color{dkgreen},       % comment style
  stringstyle=\color{mauve},         % string literal style
  escapeinside={\%*}{*)},            % if you want to add LaTeX within your code
  morekeywords={*,...},              % if you want to add more keywords to the set
  deletekeywords={...}              % if you want to delete keywords from the given language
}

%Graphikerweiterungen f�r eps-Graphiken
\usepackage{subfigure}
\usepackage{sty/floatfig}
%\usepackage{thumbpdf}
\usepackage{epsfig}
\usepackage{rotating}

\setlength{\parindent}{0.0cm}				%Abstand bei Absatzeinr�cken (falls unerw�nscht, dann \noident)

%\usepackage[numbers,sort&compress]{natbib}           % Einfache Anpassung der Bibliographie und Zitierstile
%\usepackage{eurosym}             % Euro-Symbol (offiziell)
%
% Type 1 Fonts f�r bessere darstellung in PDF verwenden.
%
\usepackage{courier}            % Courier als \ttdefault
%\usepackage{}    				
% skalierte Helvetica als \sfdefault [scaled=1] (avant,helvet(mit scaled=...), bookman, utopia, charter, pifont, newcent)


\setkomafont{sectioning}{\normalfont\bfseries}
\setkomafont{captionlabel}{\normalfont\bfseries}
%\setkomafont{pagehead}{\normalfont\itshape}
\setkomafont{descriptionlabel}{\normalfont\bfseries}

% Caption-Stil
% \setlength{\captionindent}{3cm}
% \renewcommand{\captionlabelfont}{\bfseries \sffamily}

% Tabellen:

% gr\"{o}{\ss}ere Zeilenh\"{o}he
%\setlength{\extrarowheight}{0.2cm}

% Neue Spaltenstile, f\"{u}r Dezimalzahlen
\newcolumntype{1}{D{.}{.}{1.13}}
% und grau hinterlegte Tabellenzellen
\newcolumntype{G}{>{\columncolor[gray]{0.8}}c}

% Abk\"{u}rzung f\"{u}r graue Tabellenzelle
\newcommand{\GS}[1]{\multicolumn{1}{G}{#1}}

% Farbe zwischen Doppellinien aus \hhline:
\doublerulesepcolor{white}


\usepackage{txfonts}
\usepackage[scaled=.92]{helvet}
\usepackage{courier}


% Tabelleneinbindung
\usepackage{array}
\usepackage{float}

	%1,5 Zeilenabstand
  \renewcommand{\baselinestretch}{1.1}

	%Seite einrichten
	%\textwidth15cm
	%\topmargin0pt
	%\oddsidemargin0.86cm
	%\evensidemargin0.06cm
	%\headheight0.7cm
	%\headsep1cm
	%\textheight22.7cm
	%\footskip0.7cm
	
	%Kopf und Fusszeile einrichten
%	\markright{\thesection}
%	\pagestyle{headings}
%\setlength{\oddsidemargin}{0.46cm}
%\setlength{\textwidth}{16cm}


% Mehr Platz f�r Bilder auf den Seiten:
\renewcommand{\topfraction}{.95}
\renewcommand{\bottomfraction}{.95}
\renewcommand{\textfraction}{.05}

% EPS-Grafiken:
\graphicspath{{images/}}
% \setlength{\intextsep}{5ex plus 1ex minus 1ex}

% Erstellung von Index
\usepackage{makeidx}
\makeindex
\definecolor{ForestGreen}{rgb}{0, 0.545, 0} %definiert dunkleres gr�n mit name in rgb (werte zwischen 0 und 1)
\usepackage{hyphenat} %Silbentrennung verhindern mit: \nohyphens{***}

\usepackage{wrapfig}

%\usepackage[natbib=true,style=numeric]{biblatex}

\usepackage[babel]{csquotes}
\bibliography{lit}  % bindet lit.bib ein
\usepackage{hyperref}  % unbedingt als letztes Paket laden
