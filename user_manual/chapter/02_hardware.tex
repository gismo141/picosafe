\chapter{Hardware}

\section{Introduction}

The picosafe hardware is small, has a low power consumption, but provides
enough performance for many applications.

\section{Features}

\begin{itemize}
\item LPC3143:
  \begin{itemize}
  \item 270 MHz, 32bit  ARM926EJ-S
  \item 16 kB D-cache and 16 kB I-cache
  \item memory management unit
  \item 192 kB embedded SRAM
  \item AES decryption unit
  \item secure one-time programmable memory for AES key
  \item 128 bit unique id
  \item random number generator
  \item High-speed USB 2.0 (OTG, Host, Device) with on-chip PHY
  \item DMA controler
  \end{itemize}
\item 32 MB SDRAM
\item SD-card
\item two LEDs (green and red)
\end{itemize}

\section{LPC3143}

The LPC3143 provides a 32-bit ARM926EJ-S with 270 MHz, 192 kB internal SRAM and
USB 2.0 (OTG, host, device).

The security features include a random number generator, an AES decryption
engine and a secure one-time programable (OTP) memory for AES key storage. The
processor also supports running encrypted code after reset. This option is
turned on by default and there is no possibility to run any code after reset
without knowledgte of the correct AES key stored in the OTP area of the
processor. For security reasons JTAG is completely disabled.

After reset the AES key may be read from the internal SRAM. However, it is
possible to protect the AES key from being read by writing an appropriate value
to the rprot (read protection) register. A full description about the OTP
memory and read protection is given in the LPC314x user manual in chapter 19.

\section{Peripherals}

Picosafe contains two LEDs and a USB interface:
\begin{itemize}
\item A green LED that is connected with the input voltage and will be turned on when the device is supplied with power,
\item and a red LED that can be turned on and off by software (GPIO3).
\item USB device interface
\end{itemize}
