\chapter{initramfs}

\section{Introduction}

An initramfs (initial ram filesystem) is a compressed archive that contains all
needed files and programs to boot a linux system. The kernel mounts the content
of the initramfs to the root-directory and starts a program (typically
\texttt{/sbin/init} or \texttt{/init}) that mounts the real root-filesystem and
starts the linux system. 

In order to start up fast the initramfs should be small, therefore tools like
busybox are used.

\section{Creating initramfs}

In order to create a initramfs, you may use the script \texttt{geninitramfs.sh}. The
script performs following steps:

\begin{enumerate}
\item Checking, if the cross compiler is installed correctly.
\item Creating the directory structure for the initramfs.
\item Creating device nodes like \texttt{/dev/null} or \texttt{/dev/console}.
\item Copying busybox to the initramfs. The script will compile busybox first if needed.
\item Installing all Debian packages in the directory \texttt{packages/}.
\item Copying kernel modules. The script will compile the kernel modules first if needed.
\item Copying the content of the directory \texttt{initramfs-root/} to the initramfs.
\end{enumerate}

To create the initramfs, change to the directory \texttt{initramfs/} and
execute the script \texttt{geninitramfs.sh}. This script will need root
priviledges in order to create device nodes.

\texttt{\$ cd initramfs/} \\
\texttt{\$ ./geninitramfs.sh} \\
\texttt{Deleting old initramfs directory... } \\
\texttt{Creating directory structure... } \\
\texttt{Creating device nodes... } \\
\texttt{Compiling busybox not needed (skipped). } \\
\texttt{Creating important symlinks... } \\
\texttt{Copying busybox to initramfs... } \\
\texttt{Installing packages: } \\
\texttt{cryptsetup-udeb\_1.1.3-4squeeze2\_armel.udeb... } \\
\texttt{libc6-udeb\_2.11.3-4\_armel.udeb... } \\
\texttt{libdevmapper1.02.1-udeb\_1.02.48-5\_armel.udeb... } \\
\texttt{libgcc1\_4.4.5-8\_armel.deb... } \\
\texttt{libncurses5\_5.7+20100313-5\_armel.deb... } \\
\texttt{libpopt0-udeb\_1.16-1\_armel.udeb... } \\
\texttt{libreadline6\_6.1-3\_armel.deb... } \\
\texttt{libuuid1-udeb\_2.17.2-9\_armel.udeb... } \\
\texttt{lua5.1\_5.1.4-5\_armel.deb... } \\
\texttt{readline-common\_6.1-3\_all.deb... } \\
\texttt{Installing kernel modules... } \\
\texttt{Copying files into initramfs... } \\
\texttt{./init... } \\
\texttt{./var/www/index.html... } \\
\texttt{./var/www/cgi-bin/password.lua... } \\
\texttt{./etc/httpd.conf... } \\
\texttt{./etc/udhcpd.conf... } \\
\texttt{Done.}

The initramfs will be created in the directory \texttt{output/}.

\section{Configuring initramfs}

You may change the initramfs to meet your demands. This may include adding or
removing kernel modules, installing further programs or changing the
\texttt{/init} program.

\subsection{Adding and removing kernel modules}
To add or remove kernel modules, open the script \texttt{geninitramfs.sh} and
adjust the variable \texttt{KERNELMODULES}. This variable is a array that
stores all kernel modules that should be copied to the initramfs.

\subsection{Busybox}
Busybox provides most of the standard unix tools like \texttt{cp} or
\texttt{ln}, including a shell. You may adjust the configuration of busybox and
add or remove features.

To configure busybox, change to the directory \texttt{initramfs/busybox} and
run \texttt{make menuconfig}. If you want to use the default configuration,
copy the file \texttt{config\_busybox} to the current directory. Make sure the
path for the cross compiler is set correctly.

\texttt{\$ . /opt/eldk-5.2.1/picosafe.sh} \\
\texttt{\$ cd initramfs/busybox} \\
\texttt{\$ cp ../config\_busybox . \# if you whish to use the default configuration} \\
\texttt{\$ make menuconfig}

\subsection{Adding and removing programs}
If you wish to add a program or librariy to the initramfs, copy the Debian
package (for ARM) in the directory \texttt{initramfs/packages}. The script \texttt{initramfs.sh}
will install the Debian package to your initramfs. Please note that no
dependencies are resolved. So, if you want to install a package A that depends
on package B, make sure to copy the Debian packages A and B in the
directory \texttt{initramfs/packages}.

If you don't need a certain program or library on the initramfs, just remove
the packages in the directory \texttt{initramfs/packages}.

\subsection{Adding files to the initramfs}
You may need to add your own files to the initramfs. These files may be
configuration files or your own programs.

To add your own files to the initramfs, copy the files to the directory
initramfs-root. The script genintramfs.sh will copy all files in this directory
to the root-directory of the initramfs. If the file already exists on the
initramfs, the file will be overwritten.

\subsection{/init}
\texttt{/init} is the program that will be executed by the kernel with pid 1.
You may change this script to meet your demands. You may use a script or a
binary executable.

The default \texttt{/init} script will perform following steps:
\begin{enumerate}
\item Creating symlinks to busybox applets.
\item Mounting \texttt{/proc}, \texttt{/sys} and the SD-card.
\item Setting up a network connection over USB and starting a dhcp server.
\item Starting a webserver. This webserver will provide a page with a password field. The user then must insert the password for the encrypted root filesystem.
\item If the user provided the password, the encrypted root-directory is mounted and the linux system is started.
\end{enumerate}

\section{Hiawatha webserver}

Hiawatha is a lightweight webserver with SSL and CGI support.

\subsection{Compile (automatic)}

To compile hiawatha webserver for picosafe, you may use the \texttt{build.sh} script:

\begin{enumerate}
\item Make sure, you have at least version 2.8.4 of \texttt{cmake} installed.
\item Change to the directory \texttt{initramfs/hiawatha/build/}

\texttt{cd initramfs/hiawatha/build/}
\item execute the script \texttt{build.sh}

\texttt{./build.sh}
\end{enumerate}

This will compile the hiawatha webserver and also copy the hiawatha binary and
the libpolarssl to the initramfs directory.


\subsection{Compile (manual)}
To compile hiawatha webserver for picosafe, perform following steps:

\begin{enumerate}
\item Make sure, you have at least version 2.8.4 of \texttt{cmake} installed.
\item Download the latest source from \url{http://www.hiawatha-webserver.org/}.
\item Set the environment to use the cross compiler:

\texttt{. /opt/eldk-5.2.1/picosafe.sh}
\item Extract the source ball and change to the new directory:

\texttt{tar xzvf hiawatha-8.6.tar.gz; cd hiawatha-8.6/}
\item Create a directory \texttt{build/} and change to it:

\texttt{mkdir build; cd build/}
\item Create a file \texttt{crosscompile.cmake} with this content:

\begin{lstlisting}
SET(CMAKE_SYSTEM_NAME Linux)  # Tell CMake we're cross-compiling
include(CMakeForceCompiler)
# Prefix detection only works with compiler id "GNU"
# CMake will look for prefixed g++, cpp, ld, etc. automatically
CMAKE_FORCE_C_COMPILER(arm-linux-gnueabi-gcc GNU)
\end{lstlisting}

\item Run \texttt{cmake} to create the makefiles:

\texttt{cmake .. -DCMAKE\_TOOLCHAIN\_FILE=./crosscompile.cmake -DENABLE\_XSLT=off}

\item compile

\texttt{make}

\end{enumerate}


\subsection{Configuration}

If you want to use SSL, create a certificate and put it in the hiawatha configuration directory:

\texttt{openssl req -x509 -nodes -days 365 -newkey rsa:1024 -keyout mycert.pem -out hiawatha.pem}
