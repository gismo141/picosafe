\chapter{Linux Kernel}

\section{Introduction}

The linux kernel provides abstraction layers between hardware and programs. It
performs essential sevices like memory management and CPU scheduling, and
provides APIs for accessing hardware.

\section{Configuring and Compiling}

To configure and compile the kernel, change to the directory
\texttt{kernel/linux-2.6.33-lpc313x} and run \texttt{make menuconfig} to change the
configuration, and then run make zImage and make modules. Make sure the path to
your cross compile is set correctly. If you want to use a initramfs, make sure
that the directory with the initramfs is set up correctly. In general, you need
to run geninitramfs.sh before compiling the kernel. See the chapter about
initramfs.

\texttt{\$ . /opt/eldk-5.2.1/picosafe.sh} \\
\texttt{\$ cd kernel/linux-2.6.33-lpc313x/} \\
\texttt{\$ make menuconfig} \\
\texttt{\$ make zImage} \\
\texttt{\$ make modules}

This will compile the linux kernel (\texttt{make zImage}) and the kernel modules (\texttt{make modules}). Your compiled kernel will be stored in the file \texttt{arch/arm/boot/zImage}.

\section{Hardware Random Number Generator}

It is possible to access the hardware random number generator of the LPC3143
from userspace.  To do so, you will need the driver lpc314x-rng (Device Drivers
$\rightarrow$ Character Devices). If the driver is compiled as a module, make sure
you have loaded the module:

\texttt{\$ modprobe lpc314x-rng}

To read data from the random number generator you just need to read the file
\texttt{/dev/hwrandom}. If this device file does not exist, you may create it
using \texttt{mknod}:

\texttt{\$ mknod /dev/hwrandom c 10 183}

\section{LED}

There is a driver available to control picosafe's LED. The driver is called
\texttt{leds-picosafe} (Device Drivers $\rightarrow$ LED Support).

If the driver is loaded, you can access the LED through the \texttt{/sys}
filesystem.

Examples:
\begin{itemize}
\item Turn LED on:

\texttt{echo 1 > /sys/class/leds/picosafe::blue/brightness}
\item Turn LED off:

\texttt{echo 0 > /sys/class/leds/picosafe::blue/brightness}

\item Use heartbeat trigger:

\texttt{echo heartbeat > /sys/class/leds/picosafe::blue/trigger}

\item Show available triggers:

\texttt{cat /sys/class/leds/picosafe::blue/trigger}

\item Disable any trigger:

\texttt{echo none > /sys/class/leds/picosafe::blue/trigger}
\end{itemize}

% \section{zram}
% 
% zRam creates a compressed block device in RAM that can be used for swapping.
% This way zram may help to increase performance by avoiding to swap on a slow
% disk. 
% 
% The block device is usually \texttt{/dev/zramX} and can be created using
% \texttt{mknod}:
% 
% \texttt{\$ mknod /dev/zram0 b 253 0}
% 
% To use zram, load the driver if it is a module, and specify the size of the
% block device. For example, this will create a block device of 10MB:
% 
% \texttt{\$ echo \$((10*1024*1024)) > /sys/block/zram0/disksize}
% 
% Then just call \texttt{swapon} on the block device:
% 
% \texttt{\$ swapon /dev/zram0}
