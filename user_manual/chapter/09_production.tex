\chapter{Production}
\label{production}

\section{Introduction}

This chapter describes the production of a picosafe device.

\section{Writing AES key (automatic)}

\begin{enumerate}
\item Generate a key

\texttt{\$ picosafe\_genkey\_aes keyfilename}

\item Change to the directory \texttt{bootloader/} and run

\texttt{./build.sh -a fuse -k keyfilename\_picosafe.key -o apex-fuses-fuse.dfu.rom}

This will create the file \texttt{apex-fuses-dfu.rom}.

\item Start the dfu bootloader once

\texttt{sudo dfu-util -R -t 2048 -D apex-fuses-dfu.rom}
\end{enumerate}

\section{Writing AES key (manual)}

\begin{enumerate}
\item Generate keys:

\texttt{\$ picosafe\_genkey\_aes keyfilename}

\item Connect picosafe to USB and wait for a device PHILIPPS (check \texttt{lsusb}).
\item Connect USART cable.
\item Connect to USART:

\texttt{\$ picocom -b 115200 /dev/ttyUSB0}
\item Start apex bootloader using \texttt{dfu-boot}

\texttt{\$ sudo dfu-util -R -t 2048 -D bootloader/apex-dfu.rom}

\item Blow aes key:

\texttt{apex> fuse blow\_aes\_key AES\_KEY}

\item Enable encrypted boot:

\texttt{apex> fuses aes\_enable}

\item Set security level:

\texttt{apex> fuses set\_security\_level 3}

\item Disable DFU boot:

\texttt{apex> fuses blow 502}
\end{enumerate}

\section{Creating a new root-filesystem}

This section will show you, how to save the filesystem of a picosafe stick. You
can then use the root-filesystem to create new picosafe sticks.

\begin{enumerate}
\item Open the picosafe device, extract the SD card and insert it into your pc.-
\item Mount the encrypted LUKS container.
\item Copy the root-filesystem to a local directory.
\item Change to the directory and become root.

\texttt{\$ cp -a /media/b707efce-7235-420f-b664-d92679c79b81/* rootfs/}

\item Remove the swap file. The swap file be created automatically.

\texttt{\$ rm swapfile}

\item Delete private files in the home directories \texttt{root/} and \texttt{/home/picosafe}:

\texttt{> root/.bash\_history} \\
\texttt{> root/.viminfo} \\
\texttt{rm -r root/.subversion} \\
\texttt{rm -r root/.ssh} \\
\texttt{> /home/picosafe/.bash\_history} \\
\texttt{> /home/picosafe/.viminfo} \\
\texttt{rm -r /home/picosafe/.subversion} \\
\texttt{rm -r /home/picosafe/.ssh}

Make sure to delete any private files and not only the listed ones.

\item Delte \texttt{.svn/} directories:

\texttt{\$ rm -r `find -name .svn -type d`}

\item Create tarball:

\texttt{\$ tar --preserve-permissions --preserve-order --numeric-owner -czf ../picosafe\_rootfs.tar.gz .}

\end{enumerate}
