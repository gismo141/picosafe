\chapter{Picosafe tools}
\label{tools}

\section{Introduction}

This chapter describes tools you need to encrypt and sign the kernel, so
picosafe will boot it.

\section{Compiling}

To compile the programs change to the directory \texttt{tools} and execute \texttt{make}:

\texttt{\$ cd tools} \\
\texttt{\$ make}

This will compile the tools \texttt{picosafe\_aes} \texttt{picosafe\_genkey\_rsa},
\texttt{picosafe\_genkey\_aes} and \texttt{picosafe\_sign}.

If you wish to install these tools, run

\texttt{\$ sudo make install}

to copy the files to \texttt{/usr/bin}.

\section{picosafe\_aes}
\label{picosafe_aes}

\texttt{picosafe\_aes: [OPTIONS] FILE}

This program will encrypt/decrypt FILE.

Options may be:
\begin{itemize}
\item \texttt{-k FILENAME}: use key stored in FILENAME (if omitted, will try key.txt)
\item \texttt{-o FILENAME}: write output to FILENAME (if omitted, FILE will be overwritten)
\item \texttt{-d}: don't encrypt, but decrypt FILE
\item \texttt{-h}: print help
\end{itemize}

Example: Encrypt file \texttt{/mnt/zImage} and write it to \texttt{/mnt/zImage.crypt}:

\texttt{\$ picosafe\_aes -d /mnt/zImage.crypt /mnt/zImage}


\section{picosafe\_genkey\_aes}
\label{picosafe_genkey_aes}

\texttt{picosafe\_genkey\_aes: [OPTIONS] SUFFIX}

This program will create AES key files for lpc-tools and picosafe tools. The
files created are \texttt{FILE\_picosafe.key} and \texttt{FILE\_lpc.key}.

Options may be:
\begin{itemize}
\item \texttt{-h}: print help
\end{itemize}

Example: Create a random key and store it in the files \texttt{key\_picosafe.key} and \texttt{key\_lpc.key}

\texttt{\$ picosafe\_genkey\_aes key}


\section{picosafe\_genkey\_rsa}
\label{picosafe_genkey_rsa}

\texttt{picosafe\_genkey\_rsa [outputdir]}

This program will create a RSA 1024bit key pair. The public and private keys
are written to the files \texttt{picosafe\_pub.txt}, \texttt{pubkey.h} (public
key) and \texttt{picosafe\_priv.txt} (private key).

If no output directory is given, the current directory will be used.

The private key (\texttt{picosafe\_priv.txt}) can be used to sign files using
\texttt{picosafe\_sign}.

The public key can be used to verify signatures. The file \texttt{pubkey.h} can
be used by the apex bootloader to verify the signature of a kernel.

Please read the picosafe manual on apex for further information.


\section{picosafe\_convert}
\label{picosafe_convert}

This program converts keys saved in picosafe format to the LPC format.

Options may be:
\begin{itemize}
  \item \texttt{-h, --help}: show help message and exit  
  \item \texttt{-f FILENAME, --file=FILENAME}: input key file in picosafe format
  \item \texttt{-o OUTPUT, --output=OUTPUT}: output key file in lpc format (default: stdout)
  \item \texttt{-n, --newline}: add trailing newline
\end{itemize}


\section{picosafe\_sign}
\label{picosafe_genkey_sign}

\texttt{picosafe\_sign: [OPTIONS] FILE}

This program creates a signature for file \texttt{FILE}.

The private key must be generated by \texttt{picosafe\_genkey}. If the flag
\texttt{-p} is not given the file \texttt{picosafe\_priv.txt} will be used as private key.

The program will store the signature to the file \texttt{FILE.sig}.

Options:
\begin{itemize}
  \item \texttt{-p PRIVATEKEY} path to private key
  \item \texttt{-h} show this help
\end{itemize}
